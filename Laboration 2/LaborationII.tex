\documentclass{article}
\usepackage{amssymb}
\usepackage[T1]{fontenc}
\usepackage[utf8]{inputenc}
\usepackage{xcolor}
\usepackage{color}
\usepackage{verbatim}
\usepackage{hyperref}
\usepackage{listings}
\usepackage{mathtools}

\newcounter{questioncounter}
\newcounter{exercisecounter}
\setcounter{questioncounter}{0}
\setcounter{exercisecounter}{0}


% Questions
\newcommand{\question}[1]{
  \refstepcounter{questioncounter}
  \addcontentsline{toc}{subsubsection}{Fråga: 1.\thequestioncounter{} #1}
  \vspace{1em}~
  \\\normalfont{\large{\bfseries{\hspace{0.5em}Fråga 1.\thequestioncounter \hspace{1em}#1}}}\\\\
}

% Exercises
\newcommand{\exercise}[1]{
  \refstepcounter{exercisecounter}
  \addcontentsline{toc}{subsubsection}{Uppgift: 1.\theexercisecounter{} #1}
  \vspace{1em}~
  \\\normalfont{\large{\bfseries{\hspace{0.5em}Uppgift 1.\theexercisecounter \hspace{1em}#1}}}\\\\
}

\lstset{
  language=[Sharp]C,
  basicstyle=\color[rgb]{0.3,0.3,0.3}\ttfamily,
  keywordstyle=\color[rgb]{0,0.5,0.5},
  numberstyle=\color[rgb]{0.7,0.7,0.7},
  commentstyle=\color[rgb]{0.1,0.5,0.1},
  stringstyle=\color[rgb]{0.6,0.1,0.5},
  backgroundcolor=\color[rgb]{0.95,0.95,0.95},
  showstringspaces=false,
  numbers=left,
  breaklines,
  breakatwhitespace,
}



\begin{document}

  \title{Labb II -Ordo, sortering och sökning }
  \author{ Algoritmer och datastrukturer | Uppsala Universitet | VT15 }
  \date{}
  \maketitle


  \section*{Inledning}
   I denna laboration kommer vi att behandla ordo, sortering och sökning. Det är viktigt att
   man verkligen förstår hur sorteringsalgoritmerna fungerar, så använd papper och penna!
   Att bara skriva in kod utan att förstå de mekaniska inslagen som utförs resulterar enkom i
   framtida bekymmer (bl a på tentamen). Vidare är det viktigt att förstå hur man använder
   stora ordo och att vara noga med att visa alla uträkningar. 



  \section*{Ordo - Del 1
  }

  I denna del ska ni få en känsla för programs komplexitet, dvs hur antalet operationer i ett
  program ökar med stigande input (n). För varje nedanstående kodsnutt ska ni ange dess
  Ordo. Detta kan ni göra på flertalet sätt. Genom att definiera ett polynom som beskriver
  hur komplexiteten (antal varv i looparna) beror på input (n) eller genom att för varje loop
  i programmet motivera dess Ordo och sedan ange och motivera hela programmets Ordo.
  Ange ordo för följande kodsnuttar. Redovisa alla uträkningar och motivera
  resultaten.
  
  
   \begin{lstlisting}
1. for ( i = 1 ; i < n ; i++)
		for ( j = 1 ; j < n ; j++)
    \end{lstlisting}
    
    \begin{lstlisting}     
2.	for ( i = 1 ; i < n ; i++)
		for ( j = 1 ; j < i ; j *=2) 
    \end{lstlisting}
    \begin{lstlisting}
3.	for ( i = 1 ; i * i < n ; i++)
		for ( j = 1 ; j < n ; j++) 
    for ( i = 1 ; i < n ; i++)
		for ( j = i \% 5 ; i + j < 2000; j++) 
    \end{lstlisting} 
    \begin{lstlisting}   
4.	for ( i = 0 ; i <= n ; i *=2)
    \end{lstlisting}

  
   \section*{Ordo - Del 2
    }
  
  När vi studerar tidskomplexiteten $T(n)$ för algoritmer är det väldigt sällan vi är intresserad av det exakta resultatet. Ofta är vi endast intresserade av hur $T(n)$ växer beroende av $n$. Att approximera en tid så är Ordo-notation är ett bekvämt sätt att beskriva en funktions asymptotiska beteende, det vill säga hur snabbt den växer.
  
Även om vi vet algortims ordo så kan vi inte uttala oss om hur lång tid det kommer ta utan att vi vet hur tiden kommer att växa beroende av $n$. Vi hänvisar främst till föreläsningarna om den bakomliggande teorin men ger en påminnelse om hur de olika algoritmerna kan ha för ordo. Observera att fler finns. 
  
  \begin{center}
    \begin{tabular}{| l | c |}
      \hline
      \texttt{$\mathcal{O}(c)$} & utopisk\\ \hline
      \texttt{$\mathcal{O}(log n)$} & utmärkt \\ \hline
      \texttt{$\mathcal{O}(n)$} & mycket bra\\ \hline
      \texttt{$\mathcal{O}(n log n)$} & hygglig \\ \hline
      \texttt{$\mathcal{O}(n^2)$} & inte så bra \\ \hline
      \texttt{$\mathcal{O}(c^n)$} & katastrof \\
      \hline
    \end{tabular}
  \end{center}
  Låt oss definiera följande funktioner: \\
 \begin{center}
 \begin{tabular}{ l }
	$f(n) = 0.001n^2 + n + 100$ \\
	$g(n) = 30n + 3000$\\
	$h(n) = 0.0001n^3 + 3$\\
	$q(n) = n^2 + 26n + 2988$\\\\
	\end{tabular}
 \end{center} 
  Kom ihåg att ordo är det värsta tänkbara fallet. Det finns även inversen av ordo som betecknas med stora omega $\Omega$ alltså den bästa tänkbara tiden. Viktigt att förstå att det är vi som implementerar algoritmerna som kan påverka $\mathcal{O}$ och $\Omega$. Betrakta nedanstående kod. Som den är utformad nu så är Search en algortitm är  $T(n)$ $\in$ $\Omega  (n)$ och $T(n)$ $\in$ $\mathcal{O}(n)$. Din uppgift blir helt enkelt att få algoritmen sådan att den uppfyller  $T(n)$ $\in$ $\Omega(1)$.
  \begin{lstlisting}
public bool Search(int number, int[] numbers){
     bool found = false;
     for (int i = 0; i < numbers.Length; i++)
         { 
            if(number == numbers[i])
               {
                     found = true;
               }
         }
     return found; 
   }
   \end{lstlisting}
  
   \section*{Ordo - Del 3
      }

  Ytterliggare uppgifter som är på en högre abstraktionsnivå.
  \begin{enumerate}
  
   
      \item Ponera att vi har två algoritmer $A_1$ och $A_2$ där  $A_1$ $\in$ $\Omega (n)$ och  $A_2$ $\in$ $\mathcal{O}(n^2)$ som avgör huruvida ett tal $n$ är ett primtal eller ej. Du är i behov av just en sådan algoritm. Vilken av dem bör du införskaffa givet den information du har? Motivera ditt svar.
	      
      \item Du väljer en av algoritmerna i föregående fråga och en kamrat väljer den andra. Ni
      testkör era algoritmer på en stor mängd tal. Ingen skillnad i körtid noteras. Hur är
      detta möjligt? Antag att all hårdvara och mjukvara utöver just algoritmerna som testas
      är ekvivalenta.
      
    \end{enumerate}
\section*{Sortering}
För att få förståelse för sortering och hur kostsamt det är så går denna labb ut på att experimentera hur lång den faktiska tiden är för att sortera. Vi utgår från att ni läser på teorin på Quicksort och Bubblesort och förstår hur detta görs manuellt. För uppgiften på denna labb så kommer ni inte behöva implementera detta själva utan kan använda er av koden från föreläsningarna. Ni har fått kod från föreläsningen på Quicksort och Bubblesort. Vi skall alltså ta och utföra 5 körningar av varje sortering med x st element i varje lista och ta fram den exakta körtiden för varje körning. Nedan följer en mer steg för steg beskrivning.

\begin{enumerate}
\item Börja med att skapa upp en consol-applikation i VS. I den så klistrar du in de metoder du fått från föreläsningen (dvs Bubblesort och Quicksort). Ta inte det som finns i main-metoden (om du inte vill testa din metod - då är det fritt fram).
\item Skapa en metod som retunerar en array med slumpade nummer i. Man skall när man anropar metoden kunna ange hur stor och hur många element som skall finnas i arrayen. 
\item (frivilligt) Skapa gärna en metod som kan skriva ut en array i consolen. 
\item För att kunna ta tid skall ni använda er av $Stopwath$ förklaring hur man använder denna klass finns här: http://www.dotnetperls.com/stopwatch.
\item Innan vi nu använder våra sorteringsalgoritmer så skall ni i början av Quicksort och Bubblesort starta klockan och stoppa den när den är klar med sorteringen. Sedan skall den sist i metoderna skriva ut vilken tid det tog.
\item Innan vi endast börjar kolla tiderna så kan du med fördel prova att skriva ut den osorterade arrayen för att sedan skriva ut den sorterade för att se att den gör rätt (med andra ord att du inte förstört något).
\item Nu är det dags att testa. Testa att först köra bubblesort och Quicksort på 10 element för att sedan börja utöka antal element. VARNING: Redan vid 100000 element börjar det bli slött och ni kan få vänta länge! Öka $n$ långsamt! Ni får själva söka upp IT-Avdelningen utifall datorn låser sig.
\item För varje test skriver ni ner på ett papper vilken sorteringsalgoritm och hur många element du hade och sedan hur lång tid det tog.
\end{enumerate}

Du bör nu ha fått en ganska klar analys om vad som hände och varför? Vilken algoritm var bäst? 

Det roliga stannar inte här utan nu skall vi göra ett ytterligare experiment. Från föreläsningen bör du fått veta om att Quicksort använder sig av Pivot-element. I den algoritmen som du har så tas pivot-element från mitten och nu skall du ändra den kodraden till att ta pivot-elementet längst till vänster. Alltså denna rad\\

$x = data[(l + r) / 2]; /* find pivot item */$ \\

ändrar vi till: \\\\
$x = data[l]; /* find pivot item */$ \\

(Kommentera ut din kodrad som sorterar med bubblesort) - kunde du se någon skillnad i din körning? Tog den längre tid? Lägg nu till en kodrad där du sorterar samma lista igen (alltså skickar in din sorterade lista att sorteras igen) - vad hände?

Ändra nu tillbaka till: \\

$x = data[(l + r) / 2]; /* find pivot item */$ \\

Vad händer nu? Spelar det någon roll hur vi väljer Pivot-element? Isåfall när spelar det roll? 


\section*{Sökning}
Ett vanligt problem är att man har en samling $S$, exempelvis en array, med element där man
frågar sig huruvida ett element $e$ ingår i $S$ eller ej. Skriv en funktion i C\# som tar emot en sorterad lista och utför en binär sökning. Signaturen bör således vara något likt $f(int[ ] A, int a) $ där $a$ är elementet vi söker.
\begin{enumerate}
\item Skriv en naiv lösning på sökproblemet i en array. Med "naiv" avses här en for -loop
som går igenom samtliga element från början till slut i arrayen och utför en check för
varje element. Testkör sedan funktionen på en relativt stor array du slumpat fram. Vad
är ordo?
\item Testkör din binära sökfunktion på samma array som ovan. Vad händer? Varför?
\item Sortera arrayen med quicksort eller bubble sort innan du kör den binära sökningen.
Vad är ordo nu? Motivera!
\item Ar det bättre att sortera en array och använda den binära sökningen eller att använda
den naiva lösningen direkt? Motivera!
\end{enumerate}
\end{document}