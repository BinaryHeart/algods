\documentclass{article}
\usepackage{amssymb}
\usepackage[T1]{fontenc}
\usepackage[utf8]{inputenc}
\usepackage{xcolor}
\usepackage{color}
\usepackage{verbatim}
\usepackage{hyperref}
\usepackage{listings}

\newcounter{questioncounter}
\newcounter{exercisecounter}
\setcounter{questioncounter}{0}
\setcounter{exercisecounter}{0}


% Questions
\newcommand{\question}[1]{
  \refstepcounter{questioncounter}
  \addcontentsline{toc}{subsubsection}{Fråga: 1.\thequestioncounter{} #1}
  \vspace{1em}~
  \\\normalfont{\large{\bfseries{\hspace{0.5em}Fråga 1.\thequestioncounter \hspace{1em}#1}}}\\\\
}

% Exercises
\newcommand{\exercise}[1]{
  \refstepcounter{exercisecounter}
  \addcontentsline{toc}{subsubsection}{Uppgift: 1.\theexercisecounter{} #1}
  \vspace{1em}~
  \\\normalfont{\large{\bfseries{\hspace{0.5em}Uppgift 1.\theexercisecounter \hspace{1em}#1}}}\\\\
}

\lstset{
  language=[Sharp]C,
  basicstyle=\color[rgb]{0.3,0.3,0.3}\ttfamily,
  keywordstyle=\color[rgb]{0,0.5,0.5},
  numberstyle=\color[rgb]{0.7,0.7,0.7},
  commentstyle=\color[rgb]{0.1,0.5,0.1},
  stringstyle=\color[rgb]{0.6,0.1,0.5},
  backgroundcolor=\color[rgb]{0.95,0.95,0.95},
  showstringspaces=false,
  numbers=left,
  breaklines,
  breakatwhitespace,
}



\begin{document}

  \title{Labb I -Listor, köer och stackar }
  \author{ Algoritmer och datastrukturer | Uppsala Universitet | VT15 }
  \date{}
  \maketitle


  \section*{}
   En datastruktur besitter både styrkor och svagheter. Att välja rätt datastruktur för att
   lösa ett givet problem kan därför underlätta utformningen av lösningsalgoritmen väsentligt.
   I den här laborationen kommer ni att bekanta er med olika grundläggande datastrukturer
   som listor, arrayer, köer och stackar. Tanken är att ni skall få en inblick i deras respektive
   styrkor och svagheter. Därmed bör det bli enklare att välja rätt typ av datastruktur för ett
   givet problem. Ni kommer även att implementera en egen datastruktur genom realiseringen
   av den abstrakta datatypen Påse (Bag/Multiset). Alla implementationer i denna laboration
   görs i skalprojektet som finns att tillgå på kurshemsidan. 



  \section*{Listor
  }

  En lista är en abstrakt datastruktur som kan implementeras på flera olika sätt. För er är
  antagligen List-klassen den mest bekanta sedan tidigare. Till skillnad mot en array är en
  lista inte av en förutbestämd storlek, utan storleken ändras i takt med att element läggs
  till i listan. List-klassen är dock implementerad med en underliggande array vars beteende
  ni nu skall undersöka. För att erhålla storleken av denna array används Capacity-metoden i
  List-klassen.
  
  \begin{enumerate}
    \item Skriv klart implementationen av ExamineList-metoden så att du kan genomföra din
    undersökning.
    \item När ökar listans (d v s den underliggande arrayens) kapacitet?
    \item Med hur mycket ökar kapaciteten? (undersök ett flertal utökningar!)
    \item Minskar kapaciteten när element tas bort ur listan?
  \end{enumerate}
  
  \textcolor{red}{Uppgift att fundera på: Varför ökar inte listans kapacitet i samma takt som element läggs till? Varför har inte listan 1000 platser redan från början? (DENNA SKA SKRIVAS OM!)}

  \section*{Köer}
  Datastrukturen kö (implementerad i Queue-klassen) fungerar enligt FIFO-principen (First
  In First Out), d v s att det element som ställt sig i kön först även lämnar kön först.

  \begin{enumerate}
      \item Simulera en kö med papper och penna där följande inträffar:
	      \begin{enumerate}
	          \item ICA oöpnar och kön till kassan är tom.
	          \item Kalle ställer sig i kön.
	          \item Lisa ställer sig i kön.
	          \item Kalle blir expedierad och lämnar kön.
	          \item Stina ställer sig i kön
	          \item Lisa blir expedierad och lämnar kön.
	          \item Olle ställer sig i kön.
	          \item \ldots
	      \end{enumerate}
      \item Implementera metoden TestQueue. Metoden ska simulera hur en kö fungerar genom
      att tillåta användaren att ställa element i kön (enqueue) och ta bort element ur kön
      (dequeue). Använd Queue-klassen till hjälp för att implementera metoden. Simulera
      sedan ICA-kön igen med hjälp av ditt program.
    \end{enumerate}
  \section*{Stackar}
  En stack är väldigt lik datastrukturen kö med skillnaden att en stack fungerar enligt principen FILO (First In Last Out). Alltså gäller att det element som läggs till sist till stacken (push) är det element som kommer tas bort först (pop).
  \begin{enumerate}
  	          \item Simulera ICA-kön i föregående fråga igen på papper, men använd dig av en stack istället för en kö. Varför är detta inte en strategi att föredra med avseende på vad vi försöker simulera?
  	          \item Implementera en ReverseText-metod som läser in en sträng från användaren och med hjälp av en stack vänder ordning på teckenföljden för att sedan skriva ut den omvända strängen till användaren.	          
   \end{enumerate}
  \section*{Påse (Bag) - implementation}
  Det finns ett antal datastrukturer som fungerar som samlingar av objekt. Ni har ovan bekantat
  er med listor, köer, stackar och arrayer som samtliga är exempel på dylika datastrukturer.
  Samlingar är vanliga då lagring av element i en struktur är en grundläggande förutsättning
  för databehandling. Ni skall nu göra en egen implementation av en samling i form av en
  påse. Följande operationer ska stödjas:
  
  \begin{center}
  \renewcommand{\arraystretch}{1.5}
    \begin{tabular}{ l | c | p{5cm}}
    
      
      \textbf{Namn} & \textbf{Signatur} & \textbf{Förklaring} \\ \hline
      Add & \texttt{void Add(int item)}& Lägger till item i påsen \\ 
      Remove & \texttt{bool Remove(int item)} & Tar bort item ur påsen (finns det flera med samma värde tas enbart den första bort) \\
      Clear & \texttt{void Clear()} \\ 
      Count & \texttt{int Count} & Antalet element i påsen (egenskap) \\
      Contains &\texttt{ bool Contains(int item)} & Kontrollerar om item finns i påsen  \\ 
      Items & \texttt{int[] Items()} & En array med alla element i påsen \\
      \hline
    \end{tabular}
  \end{center}
  
  Påsen ska implementeras i klassen Bag och ska vara implementerad med en array som
  underliggande datastruktur för lagring av de element som påsen innehåller. Denna arrays
  storlek ska öka och minska dynamiskt och alltid vara av samma storlek som det totala antalet
  element i påsen. Således, om vi har 15 element i påsen ska arrayen ha storlek 15 och om vi
  tar bort ett element ska den ha storlek 14 o s v.

    \begin{enumerate}
    	          \item Operationerna Add och Remove kommer att medföra att den underliggande arrayens storlek ändras. Redogör för dessa operationers utformning genom illustration på papper.
    	          \item Implementera ovanstående metoder i Bag-klassen.	  
    	          \item Skriv ett testprogram (implementera metoden \texttt{Testbag}) där man ska kunna lägga till, ta bort samt räkna element. Vidare ska man kunna undersöka huruvida ett element finns i påsen eller ej. Använda följande kommandon; add: +element, remove: -element,
    	          contains: ?element, count: !. 
    	          \\\\
    	          Exempel på testkörning:
    	          
    	          \texttt{+12}\\
    	          \texttt{Bag = 12 }\\
    	          \texttt{+3} \\
    	          \texttt{Bag =  12, 3 }
    	          \texttt{+7}\\
    	          \texttt{Bag = 12, 3, 7 }\\
    	          \texttt{-3}\\
    	          \texttt{Bag =  12, 7 }\\
    	          \texttt{?3} \\
    	          \texttt{ Contains = false} \\
    	          \texttt{!}\\
    	          \texttt{Count = 2}\\
    	          
    	              
     \end{enumerate}

  \subsection*{Abstraktion}
\textcolor{red}{Här kommer det finnas ett sätt att visa på abstraktion med hjälp av interface}



\end{document}