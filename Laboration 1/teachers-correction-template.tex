\documentclass{article}
\usepackage{amssymb}
\usepackage[T1]{fontenc}
\usepackage[utf8]{inputenc}
\usepackage{xcolor}
\usepackage{color}
\usepackage{verbatim}
\usepackage{hyperref}
\usepackage{listings}

\newcounter{questioncounter}
\newcounter{exercisecounter}
\setcounter{questioncounter}{0}
\setcounter{exercisecounter}{0}

\newlength\tindent
\setlength{\tindent}{\parindent}
\setlength{\parindent}{0pt}
\renewcommand{\indent}{\hspace*{\tindent}}


% Questions
\newcommand{\question}[1]{
  \refstepcounter{questioncounter}
  \addcontentsline{toc}{subsubsection}{Fråga: 1.\thequestioncounter{} #1}
  \vspace{1em}~
  \\\normalfont{\large{\bfseries{\hspace{0.5em}Fråga 1.\thequestioncounter \hspace{1em}#1}}}\\\\
}

% Exercises
\newcommand{\exercise}[1]{
  \refstepcounter{exercisecounter}
  \addcontentsline{toc}{subsubsection}{Uppgift: 1.\theexercisecounter{} #1}
  \vspace{1em}~
  \\\normalfont{\large{\bfseries{\hspace{0.5em}Uppgift 1.\theexercisecounter \hspace{1em}#1}}}\\\\
}

\lstset{
  language=[Sharp]C,
  basicstyle=\color[rgb]{0.3,0.3,0.3}\ttfamily,
  keywordstyle=\color[rgb]{0,0.5,0.5},
  numberstyle=\color[rgb]{0.7,0.7,0.7},
  commentstyle=\color[rgb]{0.1,0.5,0.1},
  stringstyle=\color[rgb]{0.6,0.1,0.5},
  backgroundcolor=\color[rgb]{0.95,0.95,0.95},
  showstringspaces=false,
  numbers=left,
  breaklines,
  breakatwhitespace,
}



\begin{document}

  \title{Labb II -Listor, köer och stackar }
  \author{ Algoritmer och datastrukturer | Uppsala Universitet | VT15 }
  \date{}
  \maketitle


  \section*{}
   En datastruktur besitter både styrkor och svagheter. Att välja rätt datastruktur för att
   lösa ett givet problem kan därför underlätta utformningen av lösningsalgoritmen väsentligt.
   I den här laborationen kommer ni att bekanta er med olika grundläggande datastrukturer
   som listor, arrayer, köer och stackar. Tanken är att ni skall få en inblick i deras respektive
   styrkor och svagheter. Därmed bör det bli enklare att välja rätt typ av datastruktur för ett
   givet problem. Ni kommer även att implementera en repository och med hjälp av interface demostrera styrkan med abstraktion. Alla implementationer i denna laboration
   görs i skalprojektet som finns att tillgå på kurshemsidan. I denna laboration kommer ni inte behöva lämna in svar på frågorna i texten, alltså endast kodinlämning krävs.



  \section*{Listor
  }

  En lista är en abstrakt datastruktur som kan implementeras på flera olika sätt. För er är
  antagligen List-klassen den mest bekanta sedan tidigare. Till skillnad mot en array är en
  lista inte av en förutbestämd storlek, utan storleken ändras i takt med att element läggs
  till i listan. List-klassen är dock implementerad med en underliggande array vars beteende
  ni nu skall undersöka. För att erhålla storleken av denna array används Capacity-metoden i
  List-klassen.
  
  \begin{enumerate}
    \item Skriv klart implementationen av ExamineList-metoden så att du kan genomföra din
    undersökning.
    \item När ökar listans (d v s den underliggande arrayens) kapacitet?
    \item Med hur mycket ökar kapaciteten? (undersök ett flertal utökningar!)
    \item Minskar kapaciteten när element tas bort ur listan?
  \end{enumerate}
  
  
 Uppgift att fundera på: Varför ökar inte listans kapacitet i samma takt som element läggs till? Varför har inte listan 1000 platser redan från början? - diskutera detta tillsammans med någon annan grupp. 

  \section*{Köer}
  Datastrukturen kö (implementerad i Queue-klassen) fungerar enligt FIFO-principen (First
  In First Out), d v s att det element som ställt sig i kön först även lämnar kön först.

  \begin{enumerate}
      \item Simulera en kö med papper och penna där följande inträffar:
	      \begin{enumerate}
	          \item ICA öppnar och kön till kassan är tom.
	          \item Kalle ställer sig i kön.
	          \item Lisa ställer sig i kön.
	          \item Kalle blir expedierad och lämnar kön.
	          \item Stina ställer sig i kön
	          \item Lisa blir expedierad och lämnar kön.
	          \item Olle ställer sig i kön.
	          \item \ldots
	      \end{enumerate}
      \item Implementera metoden TestQueue. Metoden ska simulera hur en kö fungerar genom
      att tillåta användaren att ställa element i kön (enqueue) och ta bort element ur kön
      (dequeue). Använd Queue-klassen till hjälp för att implementera metoden. Simulera
      sedan ICA-kön igen med hjälp av ditt program.
    \end{enumerate}
  \section*{Stackar}
  En stack är väldigt lik datastrukturen kö med skillnaden att en stack fungerar enligt principen FILO (First In Last Out). Alltså gäller att det element som läggs till sist till stacken (push) är det element som kommer tas bort först (pop).
  \begin{enumerate}
  	          \item Simulera ICA-kön i föregående fråga igen på papper, men använd dig av en stack istället för en kö. Varför är detta inte en strategi att föredra med avseende på vad vi försöker simulera?
  	          \item Implementera en ReverseText-metod som läser in en sträng från användaren och med hjälp av en stack vänder ordning på teckenföljden för att sedan skriva ut den omvända strängen till användaren.	          
   \end{enumerate}
   \newpage
  \section*{Repository (Lagringsplats) - implementation}
I denna uppgift så ska ni få bygga en så kallad "Repository" (Svenska: förvaringsplats). \\ \\ En Repository är ett mycket vanligt förekommande designmönster inom systemarkitektur och representerar en klass vars syfte är att att sköta all datahanteringslogik mot en viss datakälla (vanligtvis en Databas). \\

En Repository kan alltså ses som ett separat lager i ett system som aggerar mellanhand mellan applikationslogik och datalagringskälla. \\

Det finns många fördelar med att separera sådan typ av logik från applikationslogiken. Bland annat för att centralisera datahanteringen till ett och samma ställe. Detta leder till färre upprepningar av kod vilket i sin tur gör det 
lättare att underhålla och testa koden. \\

En Repository sköter oftast endast de mest grundläggande operationer mot datakällan. Dessa brukar kallas för CRUD-operationer som står för Create (Lagra ett objekt), Read (Hämta alla lagrade objekt), Update (Updatera ett objekt) och Delete (Tag bort ett objekt). \\

Det är just dessa operationer som eran Repository ska stödja och de finns specificerade i mer detalj nedan: \\
  
  \begin{center}
  \renewcommand{\arraystretch}{1.5}
    \begin{tabular}{ l | c | p{5cm}}
    
      
      \textbf{Namn} & \textbf{Signatur} & \textbf{Förklaring} \\ \hline
      Create & \texttt{void Create(Kontakt kontakt)}& Lägger till kontakt i lagringsplatsen\\ 
      Read & \texttt{IEnumerable<Kontakt> Read()} & Hämtar alla kontakter i lagringsplatsen \\
      Update & \texttt{bool Update(Kontakt old, Kontakt new)} & Ändrar en lagrad kontakt (om kontakten finns returnerar true annars false). \\ 
      Delete & \texttt{bool Delete(Kontakt kontakt)} & Tar bort en kontakt (om kontakten finns returnerar true annars false).\\
      \hline
    \end{tabular}
  \end{center}
  
  Som ni ser så tar metoderna i beskrivningen ovan objekt av typen Kontakt som argument. Detta är för att vi senare ska använda oss av en enkel Telefonboksapplikation för att testa denna funktionalitet.
  
  I skalprojektet finns det redan en Repository implementerad som kallas "ListRepository". ListRepository är en repository som använder en Lista som datalagringsplats. Den repository som vi ska skapa nu ska i stället använda en Array för att lagra data och vara implementerad på ett sätt som skiljer sig aningen från hur en List fungerar.
  
  Er repository ska implementeras i klassen ArrayRepository som även den finns i skalprojektet och ska alltså använda en Array som underliggande datastruktur. Till skillnad från hur en List fungerar så ska storleken av er array förändras i takt med att objekt läggs till och tas bort. Arrayen ska alltså alltid ha samma storlek som antalet lagrade element.

I Skalprojektet finns det en redan implementerad applikation "ConsoleTelefonbok" vars syfte är att testa eran repository. ConsoleTelefonbok är implementerad på så sätt att den kräver att en ArrayRepository skickas in i konstruktorn i samband med att en instans av Telefonboken skapas. Detta görs i metoden TestRepository och ni kan låta det vara så för nu.
    \begin{enumerate}
                 \item Testa att köra Telefonbokapplikationen i dess befintliga skick genom att köra programmet och navigera till Repository valet i programmets meny och följ applikationens egen meny. Testa samtliga operationer.
    	          \item Implementera sedan samtliga CRUD-metoder som beskrevs på föregående sida i ArrayRepository klassen.
    	          \item När det är klart så ska ni ändra så att ConsoleTelefonbok tar ett obejt av typen ArrayRepository som argument i konstruktorn i stället för en ListRepository (Detta kräver även att variabeln ``\_repo'' ändrar typ till ArrayRepository).
    	          \item Testa nu Telefonbok applikationen igen. Om ni gjort rätt bör den fungera precis som tidigare men nu använder den en helt annan datastruktur för att lagra Kontakt objekten.
    	              
     \end{enumerate}

  \subsection*{IRepository<T> - Abstraktion}

I den föregående uppgiften skapade ni en ArrayRepository som ni testade genom att låta den ersätta ListRepository som ConsoleTelefonbok-applikationen tidigare använde sig av. Detta var möjligt eftersom ListRepository och ArrayRepository båda stödjer precis samma operationer (De implementerade samma metodsignaturer).\\ \\
Det är ju självklart en stor fördel att kunna byta datalagringskälla på det här viset men detta krävde också att vi gick in och ändrade i implementationen av hela Telefonboksapplikationen vilket ju inte är särskilt smidigt och inte alls är möjligt för en vanlig användare av Telefonboken.

Är det möjligt att lösa detta problem? Varför ska Telefonboksapplikationen egentligen behöva bry sig om hur datalagringsklassen väljer att sköta sina operationer så länge den vet vilka operationer som stödjs? \\ \\Svaret är att den faktiskt inte behöver det, inte om vi använder oss av Abstraktion, närmare bestämt av ett Interface.

Ett Interface är en speciell typ av struktur inom programmering likt en klass men med skillnaden att den inte får innehålla någon implementation utan endast metodsignaturer. Interfaces fungerar nämligen endast som en form av kontrakt som bestämmer vilka operationer som måste finnas implementerade hos de klasser som implementerar interfacet. Just därför kallas de även ofta för Contracts. Ett interface representerar alltså en typ av abstraktion och kan liknas vid vilket annat abstrakt begrepp som helst. En "bil" är ju t.ex ett abstrakt begrepp som kan representeras av en mer specifik (konkret) sorts bil  så som en Volvo V40.

Er uppgift är nu att öka abstraktionsnivån på ArrayRepository och ListRepository och även öka flexibiliteten hos ConsoleTelefonbok så att den inte bryr sig om vad för specifik sorts repository den tilldelas.

    \begin{enumerate}
                 \item Skapa ett nytt Interface som ni kallar IKontaktRepository och låt det innehålla metodsignaturerna för CRUD operationerna som beskrevs i förra uppgiften.
                 \item Låt sedan både ArrayRepository och ListRepository implementera detta interface.
                 \item Ändra sedan ConsoleTelefonbok så att den nu accepterar en repository av den abstrakta typen IKontaktRepository.
                 \item I metoden TestRepository i Program.cs kan ni nu deklarera en variabel av typen IKontaktRepository som ni tilldelar antingen en ny ArrayRepository eller en ny ListRepository och notera att båda nu kan representeras som samma objektstyp (Volvo V40 och Saab 9000 kan båda representera en Bil).
                 \item Ge nu Telefonboken en av dessa repositories och testa så att det fortfarande fungerar. Byt sedan till den andra repositoryn och notera att det nu går att byta datakälla utan några som helst besvär.
    	              
     \end{enumerate}
     
Nu när vi kommit såhär långt har vi ökat flexibiliteten rejält hos vår Telefonbokapplikation! Våra Repositories kan dock endast hantera objekt av typen Kontakt vilket innebär att de endast kan utnyttjas av applikationer som vill lagra Kontakter. Om vi nu skulle vilja skapa ett Bibliotekssystem som bland annat lagrar böcker så skulle vi inte längre kunna använda våra repositories.\\

För att göra detta möjligt så kan vi göra våra Repositories "Generiska" (Generic).\\

En stor fördel med generiska datastrukturer så som List<T> (där T är datatypen), är att vid iteration över dess element så kan vi vara säkra på vad varje element stödjer för operationer. T.ex så kan vi med en List<int> iterera över listan och göra matematiska operationer på elementen utan att oroa oss om att något element istället är en String vilket skulle orsaka problem. Om vi itererar över en icke-generisk lista så kan vi få exceptions under run-time (Programmet kraschar när det körs) men om vi skulle försökt utföra matematiska operationer på en generisk lista av String-objekt så skulle vi istället få ett compile-time error (Error som tvingar oss att rätta till felet innan programmet kan köras).

Precis som en List eller en Array kan skapas för att hålla en viss datatyp
(List<int>, List<Kontakt>, int[], Kontakt[]) så kan vi göra så att våra Repositories göra samma sak.

    \begin{enumerate}
                 \item Lägg till "<T>" efter interface deklarationen av IKontaktRepository så att det blir IKontaktRepository<T> och ändra även alla förekomster av objekttypen Kontakt i Interfacet till "T" (Utan parenteser). Eftersom operationerna specificerade i ITelefonbokRepository nu kan användas för andra objekttyper än Kontakt så kan ni nu byta namn på Interfacet till endast "IRepository<T>"
    	              \item Gör nu även ListRepository och ArrayRepository generiska på samma vis.
     \end{enumerate}

Testa nu att  skapa en ArrayRepository i TestRepository-metoden som hanterar objekt av en annan typ än Kontakt. Vi kan nu inte längre använda vår Telefonbok för att testa våran repository eftersom den inte är byggd för något annat än Kontakter men ni kan manuellt testa att lägga till objekt av den valde objekttypen genom att anropa Create metoden.

Reflektera slutligen över den ökade flexibiliteten som Interfaces och Generics kan erbjuda. Tänk på att era repositories lika gärna hade kunnat använt en databas respektive en web service som datalagring, (DatabaseRepository, WebServiceRepository) och att utbytet av repository till och med skulle kunna ske under run-time!
\end{document}